\section{Equations and graphs of Circles, Ellipses and Hyperbolas}\label{sec:equations-and-graphs-of-circles,-ellipses-and-hyperbolas}

\subsection{Circles}\label{subsec:circles}

\subsubsection{Standard equation of a circle:}
$\sqrt{(x-h)^2 + (y-k)^2}= r$ \\ \\
Where $(h,k)$ is the center and $r$ is the radius.

\subsubsection{General equation of a circle:}
$x^2 + y^2 + Dx + Ex + F = 0$ \\ \\
To go from general equation to standard equation, complete the square for both $x$ and $y$.

\subsubsection{Finding equation of circle from points:}
Recall that the standard equation for a circle is \\ \\
$\sqrt{(x-h)^2 + (y-k)^2}= r$ \\ \\
All points on the circle follow this equation.
By substituting $x$ and $y$ for 3 points on the circle, we can solve for $h$ and $k$.
3 points are required to uniquely define a circle.

\subsection{Ellipses}\label{subsec:ellipses}

\subsubsection{Standard equation of an ellipse:}
$\frac{x^2}{a^2}+ \frac{y^2}{b^2}= 1$ \\ \\
Where $-a$ and $a$ are the x-intercepts and $-b$ and $b$ are the y-intercepts for an ellipse centered at the origin.

\subsubsection{Drawing an ellipse}
1. Transform equation into standard form (divide such that RHS = 1) \\
2. Mark x and y intercepts \\
3. Connect the dots

\subsection{Hyperbolas}\label{subsec:hyperbolas}

\subsubsection{Types of hyperbolas:}
Type I (positive $x^2$): $\frac{x^2}{a^2}- \frac{y^2}{b^2}= 1$ \\
Type II (positive $y^2$): $\frac{y^2}{b^2}- \frac{x^2}{a^2}= 1$ \\ \\
Type I hyperbolas are "horizontal".
They cut the x axis at $-a$ and $a$. \\ \\
Type II hyperbolas are "vertical".
They cut the y-axis at at $-b$ and $b$. \\ \\

\subsubsection{Finding oblique asymptotes of hyperbola:}
$y = \pm \frac{bx}{a}$

\section{Basic transformations}\label{sec:basic-transformations}
\subsection{Translation and scaling}\label{subsec:translation}
\subsubsection{Parallel to y-axis}
\textbf{Scaling comes first.} \\
Official phrasing: \\
Translation parallel to y-axis in positive/negative direction. \\
Scaling parallel to y-axis with scale factor a. \\
\subsubsection{Parallel to x-axis}
\textbf{Translation comes first.} \\
Note that directions are reversed. \\
Official phrasing: \\
Translation parallel to x-axis in positive/negative direction. \\
Scaling parallel to x-axis with scale factor $\frac{1}{a}$. \\
